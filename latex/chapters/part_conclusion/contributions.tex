\chapter{Contributions}\label{ch:contributions}
Documented in this dissertation are several novel methods for applying Deep Learning to mobile communication systems. The scope of the dissertation has been on the physical layer of the cellular domain, which means the proposed \gls{dl} methods have been applied at data related to radio propagation statistics. Deep Learning models dictates large and raw quantities of data. Radio propagation statistics possess these properties, presenting them suitable for \gls{dl}-based solutions. 

Current \acrfull{enb}, and future \gls{gnb} already utilize radio propagation statistics for many purposes; for instance the action of channel estimation as highlighted in chapter \ref{ch:channel_estimation}. Traditional methods make use of this data through engineered algorithms and solutions. However, as shown throughout this dissertation \gls{dl} is highly effective at processing raw \gls{csi} for converting and translating the otherwise unsensible channel statistics into something tangible. \gls{dl}-based solutions enable the processing of raw \gls{csi}, providing simplification and improvement of existing processes dependent on channel statistics.

This chapter summarises the contributions provided in this dissertation. The remainder of the chapter discusses the current and future applications of Deep Learning in cellular and mobile networks, primarily based on the efforts produced throughout the PhD project behind this dissertation. The source code for many of the documented methods can be found in a single repository \cite{Thrane2020RepositoryLearning}.

\section{Dissertation contributions}

Outlining the contributions of the dissertation results in the following specific items

\begin{itemize}
    \item Chapter \ref{ch:channelmodellingbasics}, No path loss performance gain is found using complex ray-tracing compared to existing empirical models.
    \item Chapter \ref{ch:satelliteImages}, Deep Learning can significantly improve path loss estimation for unseen locations
    \item Chapter \ref{ch:satelliteImages}, Geographical Images contain information useful for path loss prediction that can be extracted with supervised learning
    \item Chapter \ref{ch:deepindoor}, Deep indoor propagation characteristic is shown to be determined by a complex combination of geo-statistical features
    \item Chapter \ref{ch:deepindoor}, Deep Learning is identified as an essential tool for engineering features used for modelling Deep Indoor propagation.
    \item Chapter \ref{ch:channel_estimation}, Channel estimation in uplink transmission can significantly be improved by using Deep Learning.
    \item Chapter \ref{ch:channel_q_learning}, Deep Reinforcement Learning can enable autonomous solutions for designing optimum pilot placement in uplink.
\end{itemize}

In addition to these contributions, efforts have been spent in an attempt to outline the fundamental principles of the theory behind the methodologies behind both \acrfull{ml} and \acrfull{dl} (see chapter \ref{ch:mlbasics}), but also wireless channel models for mobile communication (see chapter \ref{ch:channelmodellingbasics}). Additionally, a description of reference signals used in both \gls{lte} and \gls{nr} networks for uplink signals have been provided (see chapter \ref{ch:pilot_sequence})


\section{Optimization procedures}
The contributions can furthermore be discussed in general terms of optimisation application. While Deep Learning is a powerful tool, the methodologies associated with the learning procedures require a significant interpretation when discussing the feasibility of the solution. It is thus essential to summarise not only the provided gains of the solutions but also the noticeable challenges - mainly associated with computational complexity and run-time hereof. Optimising is the act of effectively improving a process without the violation of constraints. In mobile communication networks, such constraints are many but primarily related to two items, time and memory. Thus, it is crucial to outline the contributions from applications and the practical constraints of mobile communication systems.

The dissertation introduces the optimisation of future cellular networks. Thus it is necessary to discuss the optimisation properties provided by the proposed Deep Learning methods, furthermore, to infer and absorb general knowledge that can contribute to future novel solutions for improving cellular networks.

The contributions can be grouped into the properties of the resulting optimisation approach. In cellular networks, and the maintenance of which, the optimisation is achievable at many levels. We can group the operational complexity of the significant contributions into offline, quasi-real-time and real-time applications. More specifically, the underlying complexity of the contributed Deep Learning-based solutions is feasible for a particular set of optimisation procedures. We define offline optimisation procedures as the use of algorithms or models where the run-time and availability is not required to be constrained in time. Examples of this can be empirical path loss models for use in the planning phase for cellular base station deployment. Of course, it would be pleasant if the computational complexity is within some timing constraints, as to not delay the planning for too long. Quasi-real-time applications do consider timing constraints but are not stringent, unlike real-time applications and operations. 

For instance, utilising Deep Learning for path loss estimation, as seen in Chapter \ref{ch:satelliteImages} offer accurate predictions in an unseen scenario with low data complexity but also a low model complexity. It is expected that autonomous driving and cognitive network is dependent on low complexity models that are accurate. Thus the contribution here is two-fold, 1) an improvement to path loss prediction can be engineered by using Deep Learning and 2) It does not require unfeasible computational complexity to do so. In short, the contributions of Chapter \ref{ch:satelliteImages} is not only related to the obtained accurate path loss estimations but also the application in which such a solution can be utilised. It is believed, and shown, that Deep Learning is capable of enabling future novel solutions requiring high accuracy path loss estimations operating within real-time or at least quasi-real-time constraints. 

Improving channel estimation is a real-time application due to immediate changes to channel statistics. It could be argued that in some cases of stationary users, the changes to channel statistics only require quasi-real-time optimisation. However, regardless, it is considered a requirement of the optimisation procedure that it can be completed in real-time. The contributions of the chapter \ref{ch:channel_estimation} shows that this is to some extent, possible but with a few apparent caveats. Changes in fundamental channel characteristics cause immediate changes in the channel statistics, which is unexplored for the documented deep learning-based channel estimator. The contribution is limited to the initial exploration and performance comparison of the using \gls{dnn} methodologies, of which the performance gain is significant compared to traditional methods. Thus the contribution is limited to the application of channel estimation for stationary or slow-moving \glspl{ue}. Additional work is required for analysing the practical feasibility of both trained and untrained \gls{dnn} in channel estimation solutions.

In \ref{ch:channel_q_learning} the area of optimising channel estimation is attacked from the point of strategically placing pilots to exploit the channel statistics. Unlike the deep channel estimator, the improvements are not gained by improving traditional channel estimators, but rather interacting in such a way with the environment that exploits the observed channel statistics. 


\section{Challenges}
Several different implementations have been completed and documented throughout this thesis. During the development, implementation and testing of the methods, many challenges have been identified. These challenges are essential contributions to put forward for the future development of the proposed methods or related approaches. The practical challenges have been attempted discussed throughout each chapter of this thesis. The result of this discussion is vital for identifying and concluding overall challenges for \gls{ml}-based solutions in mobile communication networks.


\subsection{Interpretation}
As identified from chapter \ref{ch:satelliteImages} (See Section \ref{sec:identified_challenges_satellite}) the use of convolutional layers on geographical images is effective for estimating path loss, but are difficult to interpret. Generally, the methods used in wireless communication are traditionally based on known theory and well-studied techniques, methods that through mathematics, can be proven and fully understood. This transparency is not entirely there yet for \gls{dl}-based solutions \cite{Samek2017ExplainableModels} but must be said to be a  requirement for any technique used in systems such as mobile communication systems to ensure reliability. It is thus important to know when and if given techniques and solutions break in order to engineer reliable systems. This challenge is also identified through the methods of Chapter \ref{ch:channel_estimation} and \ref{ch:channel_q_learning}. 

\subsection{Implementation}
As noted by the key findings of Chapter \ref{ch:channel_q_learning} and the use of Deep Reinforcement Learning, a severe bottleneck for future development and research was identified as being related to the core implementation. The implementation challenges were primarily related to the mobile network environment. In order to effectively learn, the environment of interaction needs to be as realistic as possible. Improved implementations are critical to more experiments and better hyper-parameters. By reducing the time required for training the \gls{dl}, more time can be spent on evaluating the capabilities of the proposed architecture and the required hyper-parameters. The training time is intertwined with the challenges of obtaining the best performing model complexity for generalising the learning problem. The model complexity is adjusted through the use of hyper-parameters. Finding the right hyper-parameters is a time consuming and exhaustive task, and thus an identified challenge that is of increased importance for future \gls{dl}-enabled solutions.

\subsection{Training data}
The majority of the proposed \gls{ml} solutions presented in this thesis have been a \emph{supervised} algorithm, thus learning the mapping between inputs and outputs have been the learning objective. With the exception of Chapter \ref{ch:channel_q_learning} where Deep Reinforcement Learning is utilized. Supervised solutions suffer from data acquisition, as a training set is required. Moreover, a test set is also required to ensure generalisation is partly achieved. For some applications, this may be simple to obtain however as seen for the supervised deep channel estimator in Chapter \ref{ch:channel_estimation} it is a massive problem for the practical considerations of such a solution. Deep Reinforcement Learning is a suitable approach if given limited training data, as the problem can still be formalised in a supervised manner. Thus, the identified challenge here is how to effectively determine the practical feasibility of obtaining a training set, as it will determine the usefulness of the learned method. 

\section{Deep Learning in Cellular Networks}

In chapter \ref{ch:monster}, the question \emph{"Is Deep Learning applicable to Mobile Communication systems?"} was posed. As shown by the content of the dissertation, \acrlong{dl} is a powerful tool for optimisation and solving complex tasks. The methods shown in this dissertation have demonstrated tasks in mobile communication systems where \gls{dl} is applicable, and not at the cost of increased computational complexity. 

%Deep Learning is a powerful tool for optimisation and solving complex tasks. As shown by the content of the thesis, %such solutions do not necessarily suffer from significant computational complexity.

The stringent requirements of time and memory in \gls{lte} and \gls{nr} mobile communication is crucial to abide by. The contributions of this dissertation have successfully demonstrated that such systems can coexist to further optimise complex tasks related to the physical layer of mobile communication. Next-generation communication systems are faced with even more extensive lists of constraints and requirements which increases the difficulties of engineering optimised solutions. As found throughout literature, and demonstrated in this dissertation - Deep Learning tools are expected to play a vital role in the engineering of future solutions. If not explicitly, then implicitly through automatized learning of relevant features in the massive storage of radio measurement data. The design of an effective communication system is ultimately reduced to the fundamental understanding of the propagation channel. Any additional information that can be squeezed out of channel statistics can consequently aid the communication system in increasing coverage, capacity and reliability.

So in general terms, what has been learned by applying Deep Learning to issues related to mobile communication? The complexity of a modern cellular system is staggering and will increase with the \emph{necessary} increase in capacity needs. It is shown that complex task associated with radio propagation can be improved through the use of automatized adaptive models such as \gls{dnn}. Furthermore, many computational tasks of computer vision are almost directly translatable to the physical layer of cellular systems. The complexity increase of future systems will require novel and solutions for not only improving and optimising the capacity of such systems but also in terms of management. It may be that a direct optimisation of capacity can introduce unnecessary complexity that is so difficult and complex to manage.

The application of Deep Learning has revolutionised many computational tasks; however, it is yet to revolutionise the area of mobile communication. The potential for major improvements are there, yet, it is contingent on a fundamental question of reliability. One of the identified challenges of applying \gls{dl} models is \emph{interpretation}. The cascaded architecture of complex layers learned through iterations can be difficult if not impossible, to interpret. Perceptive insight in mobile communication systems and sub-systems hereof is paramount to effective transmission, thus integrating a \gls{dl} that just \emph{works} regardless of incredible performance improvements are bound to be a reliability issue. The introduction of \emph{expert knowledge/model-aid} in chapter \ref{ch:satelliteImages} improves the interpretation properties of the model. The model is no longer tasked with learning the entirety of the problem but aided with well-known and proven theory. This approach greatly improves the interpretation as the \gls{dl} model and can show important insights into the variability offered by the model for data points not observed during training. The embedding of \emph{expert knowledge} is seen as an essential approach for the feasibility of a \gls{dl}-based application in future mobile communication systems.