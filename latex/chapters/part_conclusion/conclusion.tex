\chapter{Conclusion}\label{ch:conclusion}

This dissertation has aimed to unravel and supply answers for the question \emph{"Is Deep Learning applicable to Mobile Communication Systems?"}. Current and future mobile communication systems are becoming increasingly sophisticated in order to increase the capacity and coverage for end users. The introduction of new capacity-increasing solutions impose complexity issues that are challenging to manage and thus optimize. Therefore, novel methods for optimizing the planning and operation of mobile communication systems are vital to ensure future capacity demands can be realized.

The dissertation has presented several novel \acrlong{dl} applications for mobile communication systems. By applying techniques of computer vision and the embedding of expert knowledge to radio channel state information, complex tasks have been approximated through iterative learning and performance has been improved. Correctly, it has been shown that \acrlong{dl} is a powerful and useful tool with many practical applications on the physical layer of mobile communication systems. For instance, \acrlong{dl} has been applied for path loss estimation using additional data of geographical images and expert knowledge to boost predictive performance in unseen propagation scenarios. Additionally, the adaptive self-learning algorithms of deep reinforcement learning have been applied successfully to combat pilot contamination and reduce the overhead required for channel estimation. The results offer evidence that \acrlong{dl} is to be a necessary component for future mobile communication systems.

Also, the dissertation presents comparative studies of current traditional methodologies for coverage and capacity estimation. The results show that traditional methods with comprehensive computational complexity offer similar performance to the simplest of empirical models. Finally, it is experimentally shown that propagation models for complex deep-indoor scenarios are lacklustre and require additional research.

It can be concluded that \acrlong{dl} applies to the physical layer of Mobile Communication Systems, as impressive performance have been obtained on complex and challenging tasks. It is found that efficient implementations are paramount to the development of future \acrlong{dl}-based solutions. Efficient implementations reduce the time-consuming aspect of discovering the best model complexity and enable the exploration of generalization capabilities. 

While \acrlong{dl} has been shown to offer impressive results, it is paramount that the scope of application is adequately identified. For instance, current mobile communications systems work exceptionally well and are reliable. The results show that while significant improvements can be achieved on complex tasks, the transparency of the methods may be unforgiving in edge case scenarios. It has been shown that the interpretation of \acrlong{dl} methods can be improved by introducing expert knowledge into the training process. Doing so limits the \acrlong{dl}-based solutions in correcting the well-known, tested and reliable theory that compose the current systems. It is, therefore, a finding of the dissertation that the integration of \acrlong{dl}-solutions must initially be achieved for sub-systems where either no requirements to reliability are enforced or expert knowledge can be utilized. 

\acrlong{dl} have the capabilities to revolutionalize much of the complexity associated with modern mobile communication systems. While the resulting model structures may be convoluted and complicated, they are enabling innovative solutions that cellular systems need. If the reliability can be ensured, and the learned deep models can be understood, \acrlong{dl} can maximize the efficiency of mobile communication systems beyond traditional engineering capabilities.



% The integration of such concepts as expert knowledge or model-aided in the training of \acrlong{dl}-based solutions is a necessity to overcome transparency issues. This limits the \acrlong{dl}-based solutions to correct the well-known, tested and reliable theory that current systems are based on. 

% Novel applications of \gls{dl} in mobile communication systems must consider practical requirements. While \gls{dl} has been hailed as an essential element for future systems, it is paramount that the scope of \gls{dl}-based solutions are identified. For instance, current mobile communications systems work extremely well and are reliable systems. The integration of \gls{dl}-solutions must initially be achieved for sub-systems where no requirements to reliability is enforced. This thesis provides insight into the feasibility of \gls{dl}-enabled solutions on the physical layer of mobile communication systems. The results show that while significant improvements can be achieved on complex tasks, the associated transparency of the methods may be unforgiving in edge case scenarios. The integration of such concepts as expert knowledge or model-aided in the training of \gls{dl}-based solutions is a necessity to overcome transparency issues. This limits the \gls{dl}-based solutions to correct the well-known, tested and reliable theory that current systems are based on. 

% This have specifically resulted in advancements for the general application of \gls{dl} methods for mobile communication systems. 

% The content of this thesis has been to explore novel methods and approaches using \gls{dl} tools and methods. This thesis has resulted in several advancements for applying \gls{dl} to complex tasks in mobile communication systems. Additionally, the thesis presents the study of current traditional channel models and the resulting predictive performance. Especially, the use of computer vision techniques are highly transferable to complex problems associated with the uncertainty of the radio environment. The results offer evidence that \gls{dl} is to be a necessary component for future mobile communication systems. The results show furthermore that the physical layer is an appropriate place to achieve significant optimization gains. 


