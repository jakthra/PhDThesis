\chapter{Conclusion}\label{ch:conclusion}

The content of this thesis has been to explore novel methods and approaches using \gls{dl} tools and methods. This thesis has resulted in several advancements for applying \gls{dl} to complex tasks in mobile communication systems. Additionally, the thesis presents the study of current traditional channel models and the resulting predictive performance. 

It can be concluded that \gls{dl} is a powerful and effective tool with many practical applications on the physical layer of mobile communication systems. Especially, the use of computer vision techniques are highly transferable to complex problems associated with the uncertainty of the radio environment. The results offer evidence that \gls{dl} is to be a necessary component for future mobile communication systems. The results show furthermore that the physical layer is an appropriate place to achieve significant optimization gains. Finally, it is found that efficient implementations are paramount to the development of future \gls{dl}-based solutions. This is to reduce the time consuming aspect of discovering the best model complexity and furthermore enable the exploration of generalization capabilities. 

Novel applications of \gls{dl} in mobile communication systems must consider practical requirements. While \gls{dl} has been hailed as an essential element for future systems, it is paramount that the scope of \gls{dl}-based solutions are identified. For instance, current mobile communications systems work extremely well and are reliable systems. The integration of \gls{dl}-solutions must initially be achieved for sub-systems where no requirements to reliability is enforced. This thesis provides insight into the feasibility of \gls{dl}-enabled solutions on the physical layer of mobile communication systems. The results show that while significant improvements can be achieved on complex tasks, the associated transparency of the methods may be unforgiving in edge case scenarios. The integration of such concepts as expert knowledge or model-aided in the training of \gls{dl}-based solutions is a necessity to overcome transparency issues. This limits the \gls{dl}-based solutions to correct the well-known, tested and reliable theory that current systems are based on. 
