\chapter{Deep Learning for Radio Propagation Modelling}\label{ch:dl_radio_summary}
Radio propagation modelling is a challenging and complicated task. The purpose of a radio propagation model is to provide the most accurate estimations of signal quality metrics. However, as seen throughout this chapter, selecting the right model is a complicated procedure for any mobile communication system. The supposed performance increase provided by highly detailed ray-tracing models is expected to outperform simple empirical models. Nevertheless, as seen in chapter \ref{ch:channelmodellingbasics}, that is not necessarily the case. Empirical path loss models are known and have been shown to offer a simple and effective model for approximating received power in mobile communication systems. Additionally, such models provide useful statistics on the large-scale fading phenomenon that offer valuable margins to use in the simplest of link-budget analysis and overall communication system design. 

The data complexity associated with the use of empirical models reduces the need for any complex data engineering - this is not the case for ray-tracing solutions. In chapter \ref{ch:satelliteImages} a novel \gls{dl} method for estimating received signal \glspl{kpi} for use in mobile communication systems have been presented. The method is shown to be capable of utilizing different geographical images and features with overall low data engineering complexity and be capable of improving predictions for unseen locations, while ensuring low generalization error across different regions and data sources. Not only does this warrant a simple pipeline for estimating signal metrics, yet it also ensures that the method is sufficiently simple to use. Simplicity is a key property and enables the many applications that have made traditional empirical models incredibly useful. For instance, in greenfield deployment situations where propagation specific data may be unavailable, and therefore ray-tracing is not a possibility. By embedding expert knowledge, e.g. utilizing the knowledge provided by the empirical path loss models with \gls{dl}, an improvement of predicting signal quality metrics have been achieved.

The deployment of devices in deep-indoor areas is expected to increase with the development of such technologies as \gls{nbiot} and battery-operated sensors. Wireless channel models are essential for designing the communication infrastructure, enabling these many sensor applications. In chapter \ref{ch:deepindoor}, the challenge of modelling radio propagation in deep-indoor situations is presented. It is shown that the current empirical models available do not offer satisfactory prediction performance. Novel solutions for generating features are required. For instance, it has been shown that the geometry of underground structures is essential for estimating signal quality parameters. But, incorporating this information into existing models is an unsolved problem. In the chapter, it is identified that \gls{dl} is a promising tool for developing generalized features that are useful for estimating signal quality parameters in underground scenarios.

Radio propagation models generally consider a trade-off between performance and data complexity. Traditional modelling techniques utilizing simple empirical expressions are useful because they require simple features. \gls{dl} is capable of extending this essential property, by including data that may be simple to obtain but is exceptionally challenging to infer features from when using traditional feature engineering concepts. For instance, the meta-data that is geographical images. It is simple and easy to obtain these images but challenging to develop an algorithm so they can be used directly for improving the estimation of received signal quality parameters. \gls{dl}-based models are effective solutions for keeping data complexity of radio propagation models low while maintaining the increased predictive performance.
