\chapter*{Resumé}
Mobile kommunikationsnetværk er komplekse systemer, der er resultatet af utrolige ingeniørmæssige præstationer. Nuværende og fremtidige mobile kommunikationssystemer er - og vil i fremtiden endnu mere være - komplekse at styre og optimere. Suboptimale mobilnetværk resulterer i ineffektive implementeringer, der forårsager dårlig kundeoplevelse og øgede drifts- og kapitaludgifter for netværksoperatører. \emph{Deep Learning} har i de seneste år leveret imponerende resultater til komplekse opgaver såsom tale- og billedegenkendelse. Kompleksiteten af mobile kommunikationssytemer forventer at stige, og \emph{Deep Learning} er blevet anerkendt som en nødvendig brik til at overvinde sådanne udfordringer effektivt. 
Denne afhandling udforsker brugen af metodikker fra \emph{Deep Learning}-værktøjskassen mobile kommunikationssystemer. Endvidere bidrages med adskillige anvendelser af \emph{Deep Learning} til at løse avancerede problemer i mobile kommunikationssystemer. Hertil præsenteres tre nye metoder:

1) Nøjagtige forudsigelser af signalkvalitet ved hjælp af geografiske billeder på umålte lokalisationer med lav datakompleksitet, 2) Væsentlige forbedringer af kanalestimering opnået på sparsomme og få referencesignaler i uplink transmission, og 3) En adaptiv \emph{reinforcement}-algoritme, der er i stand til at undgå interferens i radiomiljøet samt forbedre estimeringspræsentationen af kanalen. Herudover introduceres der for empiriske undersøgelser, hvoraf de mest bemærkelsesværdige resultater opsummeres som følgende: 1) Komplekse ray-tracing-metoder viser ingen præsentationsforbedring sammenlignet med enkle, empiriske modeller i forhold til udbredelsesmodellering af radiobølger i mobile kommunikationssystemer, og 2) Nuværende modeller til dybe indendørsscenarier, såsom tunneller, viser dårlig ydeevne og kræver nye løsninger.

De præsenterede resultater i afhandlingen viser, at \emph{Deep Learning} kan opnå betydelig præsentationsforbedringer på det fysiske lag af mobile kommunikationssystemer. Effektive implementeringer af \emph{Deep Learning}-metoder samt beregningskompleksiteten heraf, er essentielle for fremtidig forskning. Disse effektivitetsforbedringer er således også en nødvendighed for at reducere det tidskrævende aspekt i at finde passende modelkompleksitet. Endeligt, så er indlejring af såkaldt \emph{expert knowledge} en nødvendighed i anvendelsen af \emph{Deep Learning} i mobile kommunikationssystemer for at sikre pålideligheden.