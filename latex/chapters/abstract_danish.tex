\chapter*{Resumé}
Mobil kommunikationsnetværk er komplekse systemer der består af utrolige ingeniørmæssige præstationer. De nuværende og fremtidige mobil kommunikationssystemer er, og i fremtiden endnu mere, komplekse at styre og optimere. Sub-optimale mobilenetværk resulterer i ineffektive implementeringer der forårsager dårlig kundeoplevelse og øgede drifts- og kapitaludgifter for netværksoperatører. Deep Learning har i de seneste år leveret imponerende resultater i komplekse opgaver så som talegenkendelse og billedegenkendelse. Kompleksiteten af mobile kommunikationssytemer forventer at stige og Deep Learning er blevet anerkendt som en nødvendig løsning til at overvinde sådanne udfordringer. Indholdet i denne afhandling er udforskning af nye metodikker fra Deep Learning værktøjskassen. Denne afhandling har bidraget med adskillige anvendelser af Deep Learning for at løse komplekse problemer i mobile kommunikationssystemer. Specifikt præsenteres der tre nye metoder.

1) Nøjagtige forudsigelser af signalkvalitet på umålte lokalisationer med lav datakompleksitet ved hjælp af geografiske billeder. 2) Væsentlige forbedringer af kanalestimering af opnået på sparsomme og få reference signaler i uplink transmission og 3) En adaptiv reinforcement indlæringsalgorithme der er i stand til at undgå interferens i radiomiljøet. Udover dette præsenteres flere empiriske undersøgelser, mest bemærkelsesværdigt kan disse resultater opsummeres som følgende. 1) Komplekse ray-tracing metoder viser ingen præsentations forbedring sammenlignet med enkle empiriske modeller til udbredelsesmodellering af radiobølger i mobile kommunikationssystemer. 2) Nuværende modeller til dybe indendørs scenarier, f.eks. tuneller viser dårlig ydevne og kræver nye løsninger.

Resultaterne præsenteret i afhandlingen viser at Deep Learning kan opnå betydelig præsetations forbedringer på det fysiske lag af mobile kommunikationssystmer. Effektive implementeringer af Deep Learning metoder og beregningskompleksisteten heraf er fundet essentielle for fremtidig forskning. Dette er således også en nødvendighed for at reducere det tidskrævende aspekt af at finde passende modelkompleksitet. Endeligt, indlejring af såkaldt \emph{expert knowledge} er en nødvendighed for anvendelsen af Deep Learning i mobile kommunikationssystemer for at sikre pålideligheden.