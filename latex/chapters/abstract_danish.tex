\chapter*{Resumé}
Mobile kommunikationsnetværk er komplekse systemer der er resultatet af utrolige ingeniørmæssige præstationer. De nuværende og fremtidige mobil kommunikationssystemer er, og i fremtiden endnu mere, komplekse at styre og optimere. Sub-optimale mobilenetværk resulterer i ineffektive implementeringer der forårsager dårlig kundeoplevelse og øgede drifts- og kapitaludgifter for netværksoperatører. \emph{Deep Learning} har i de seneste år leveret imponerende resultater til komplekse opgaver så som talegenkendelse og billede genkendelse. Kompleksiteten af mobile kommunikationssytemer forventer at stige og \emph{Deep Learning} er blevet anerkendt som en nødvendig brik til at overvinde sådanne udfordringer effektivt. Indholdet i denne afhandling er udforskning af metodikker fra \emph{Deep Learning} værktøjskassen ved brug på Mobile kommunikationssystemer. Denne afhandling har bidraget med adskillige anvendelser af \emph{Deep Learning} for at løse komplekse problemer i mobile kommunikationssystemer. Specifikt præsenteres der tre nye metoder.

1) Nøjagtige forudsigelser af signalkvalitet på umålte lokalisationer med lav datakompleksitet ved hjælp af geografiske billeder. 2) Væsentlige forbedringer af kanalestimering opnået på sparsomme og få reference signaler i uplink transmission og 3) En adaptiv \emph{reinforcement} algorithme der er i stand til at undgå interferens i radiomiljøet og forbedre kannel estimerings præsentationen. Udover dette præsenteres flere empiriske undersøgelser, mest bemærkelsesværdigt kan disse resultater opsummeres som følgende. 1) Komplekse ray-tracing metoder viser ingen præsentationsforbedring sammenlignet med enkle empiriske modeller til udbredelsesmodellering af radiobølger i mobile kommunikationssystemer. 2) Nuværende modeller til dybe indendørs scenarier, f.eks. tunneller, viser dårlig ydeevne og kræver nye løsninger.

Resultaterne præsenteret i afhandlingen viser at \emph{Deep Learning} kan opnå betydelig præsentations forbedringer på det fysiske lag af mobile kommunikationssystemer. Effektive implementeringer af \emph{Deep Learning} metoder og beregningskompleksiteten heraf er fundet essentielle for fremtidig forskning. Disse effektivitetsforbedringer er således også en nødvendighed for at reducere det tidskrævende aspekt af at finde passende modelkompleksitet. Endeligt, indlejring af såkaldt \emph{expert knowledge} er en nødvendighed for anvendelsen af \emph{Deep Learning} i mobile kommunikationssystemer for at sikre pålideligheden.