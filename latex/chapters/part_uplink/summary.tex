\chapter{Data-driven channel estimation}

Channel estimation is a core sub-system in mobile communication systems. Improving channel estimation directly improves the data rates and thus the capacity of the system. Channel estimation is a complicated field of computational efficiency and well-understood solutions. Through chapter \ref{ch:pilot_sequence} the issues of pilot contamination are highlighted and presented. It is a given that current and future mobile communication systems will suffer from pilot contamination issues due to the requirement for improved channel estimation performance. Thus, either the channel estimators needs to improve or the pilots required needs to be reduced. 

\gls{dl}-based solutions can enable accurate interpolation between sparse and noisy pilots effectively improving the channel estimation. In chapter \ref{ch:channel_estimation} a method for achieving improvements of channel estimations by using techniques from computer vision have been shown. Furthermore, by formalizing the received channel coefficients as a $2$D image, existing state-of-the-art techniques can successfully be applied. While the model is not as simple as a linear channel estimators, the scope of the problem have been reduced to uplink \gls{srs} pilots. Therefor the proposed \gls{dl}-based channel estimator and the application hereof is limited to the base station.
\gls{dl}-based channel estimators can effectively reduce the number of required pilots while improving the channel estimation accuracy. This is achieved by exploiting channel statistics in observed \gls{csi}.


The placement of pilots have a significant impact on the resulting channel estimation, regardless of what channel estimation technique is used. This is studied in chapter \ref{ch:channel_q_learning}. By introducing an intelligent actor the pilot placement can be optimized autonomously using nothing else than a raw observation of \gls{csi}. This is done by merging concepts from \gls{cnn} and reinforcement learning. The resulting deep reinforcement learning algorithm can learn the best pilot placement by interacting with the mobile environment. The adaptability of Deep Reinforcement Learning shows promising results for a completely autonomous solution. This in combination with the performance gains offered by \emph{deep channel estimators} highlights the practicality of \gls{dl}-enabled methods for solving complex issues related to channel estimation and pilot contamination.

The documented results, discussions and conclusions shows the benefits of \gls{dl}-enabled solutions for channel estimation related tasks. It can be seen that \gls{dl} techniques for computer vision are very effective at dealing with the fundamental structure of \gls{ofdm} symbols.   
