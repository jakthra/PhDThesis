\chapter*{Abstract}
Mobile Communication networks are complex systems consisting of many incredible engineering achievements. The current and future mobile communication systems are, and in the future even more so, complex to manage and optimize. This essentially results in cost ineffective deployments causing poor customer experience and increased operational and capital expenses for operators. \acrfull{dl} have in the recent years provided with impressive results in complex tasks such as speech recognition and computer vision. The complex reality of mobile communication systems is expected to increase, and \gls{dl} has been hailed as a necessary component co overcome such challenges. The content of this dissertation is exploration of novel methodologies available in \acrshort{dl} toolbox. This dissertation has resulted in several advancements for applying \gls{dl} to complex tasks in mobile communication systems. Specifically 3 novel methods are presented. 

1) Accurate signal quality predictions in unseen locations with low data complexity using geographical images. 2) Significant improvements to channel estimation applied on sparse reference signals in uplink and 3) An adaptive reinforcement learning algorithm capable of avoiding contamination in the radio environment. In addition to this, several study items are presented. Most noticeably the outcome can be summarized as, 1) Complex ray-tracing methods show little to no performance gain compared to simple empirical models for mobile communication propagation modelling. 2) Current deep indoor propagation models show poor generalized performance and require novel solutions.

The results presented throughout the dissertation are conclusive and show significant performance gains offered by \gls{dl}-based solutions on the physical layer of mobile communication systems. The implementation and thus the computational complexity of the proposed \gls{dl} methods are found to be essential and important for future research. Thus, efficient implementations are necessary to reduce the time consuming aspect of tuning model complexity which enable further discovery of the fundamental capabilities of the given method. Finally, embedding \emph{expert knowledge} into \gls{dl}-based solutions is seen as a necessity for the application in mobile communication systems to ensure reliability and transparency.